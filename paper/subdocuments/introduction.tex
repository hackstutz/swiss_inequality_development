%%%-------------------------------------------------%%%
%%% Sub document introduction %%%
%%%-------------------------------------------------%%%

\section{Introduction}



% 1.Abschnitt: Bedeutung von Ungleichheit
Economic Resources can be seen as central indicator for life chances in general and a multitude of outcomes like physical and mental health, life expectancy and crime in particular (Wilkinson and Pickett 2009). While the study of social inequality can be considered as one of the core subjects of sociology in more recent years the concern about the widening gap was addressed by global leaders (WEF 2013) and scholars alike. Empirical evidence acknowledge the supposed trend that economic inequality increased in the majority of western countries over the last decades (OECD, 2008, 2011; Gornick and Jäntti, 2013;Salverda et al. 2014). Although the rise was not uniform, a common pattern seems to be identifiable, which can be referred to as the "hollowing of the the middle class" (Alderson and Doran 2013). Households are moving towards the top and the bottom of the distribution relative to the past, which is especially problematic as the middle class can be seen as the core of western democracies or as it is stated by Stiglitz (2012,117): "our democracy is being put at peril."
\\

% Allenfalls könnte man auch (Nolan und Whealan, 2014) erwähnen. "The social Impact of income inequality: Poverty, Deprivation, and Social Cohesion" Da werden die Thesen rund um die Folgen von Ungleichheit gut zusammengefasst. 

% 2.Abschnitt: Bedeutung der Datengrundlage und Einführung von Steuerdaten
Given the importance of the subject a constant reflection about reliability of empirical data seems appropriate. Atkinson (2013:8) observes advances in technology and methodology which improves the core sources of inequality research, the household surveys.  On the other hand the labor intensive and expensive surveys around the world are subject to budget cuts and the instrument itself faces problems in form of low response rates, which affects the assessment of inequality undisputedly. These concerns have led to the search of alternative data sources, which can supplement the established survey data studies. Already Kuznet (1955) used tax data to examine the relationship between economic growth and personal distribution of income. Then it took several decades until Piketty (2001,2003) made the use of tax data fashionable again. Following his approach studies on several countries were conducted (Atkinson and Piketty 2007, 2010). Today, all existing top income tax statistics based time series are collected and accessible through the world top incomes database (Alvaredo et al. 2014). \\

% 3.Abschnitt: What do we know about Switzerland
% Offical Data Collections on development of inequality
As we focus our paper on the case of Switzerland, it is important to embed our work in the context of given publications concering inequality in income. What is known about Switzerland so far? Looking for offical data, three main sources has to be mentioned, which can be considered as de facto offical data sources: EU-SILC, HBS and LIS-data. Figure XY shows the results steming from this three sources while looking at Gini of equivalised disposable income. Up to the day, EU-SILC or Statistics on Income and Living Conditions is the main source used for policy monitoring at EU-level. The main focus of EU-SILC is to collect data on a common "framework" to ensure comparability among EU-countries and countries living around or within the EU. As a Non-EU member Switzerland implemented the instrument not from the beginning (2004) but as from 2007. Therefore this times-series doesn't cover timepoints before 2007. As graph XY shows, following the results from SILC income inequality decreased from 2007 to 2013.\footnote{Data shown in the graph was downloaded from the Eurostat Metadata-portal http://epp.eurostat.ec.europa.eu/cache/ITY_SDDS/EN/ilc_esms.htm last accessed 21.Mai 2014.} The second important source concering the distribution of income is the Household Budget Survey (HBS). The main focus of this survey lays in providing detailed data on houshold budegts. This allows researcher to look at different income concepts like income before and after public transfers. Since 2000 the survey has been conducted on a continuous basis, which allows to look at a consistend time series from 2000 to 2011. As it can be seen from graph XY the trend is rather stable.\footnote{figures shown in the graph were calculated out of the original datasets, which were kindly provided by Swiss Federal Statistical Office.} Both time-series (SILC and HBS) cover a relatively short time period. A longer period is covered in the LIS-Data-set (1982-2004). Data-provider for the LIS Data is the Swiss Federal Statistical Office too. In contrast the the aformentioned surveys the LIS-data is harmonized out of three surveys: Swiss Income and Wealth Survey (1982), Swiss Poverty Survey (1992) and the Income and Consumption survey (2000,2002,2004).  All in all the LIS dataset contains the longest time series on inequality for Switzerland. Analyzing this data Gornick and Jäntti (2013) found for Switzerland a quite substantially decreases in income inequality, contradictory to the development in most other western countries. This result is supported by Grabka and Kuhn (2012) analyzing the Swiss Household Panel (2000-2009).\footnote{A further offical database for income distribution is the OECD-Database. It includes measures from Income and Consumption survey as well. Additional data for 2008 is available from EU-Survey of Income and Living Conditions (EU-SILC). But this change in survey is considered as a strict break. Comparison before and since 2008 is not recommended (OECD 2012:315). A further important database on inequality is the GINI database which has been derived from the GINI Country Reports (Nolan et al.,2014). But this dataset doesn't cover Switzerland.}  \\


\begin{Schunk}
\begin{Sinput}
> library(ggplot2)
> library(foreign)
> number_ticks <- function(n) {function(limits) pretty(limits, n)}
> swissGini<-read.csv("data/swissGini.csv",header=TRUE,sep=";")