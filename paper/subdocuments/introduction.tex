%%%-------------------------------------------------%%%
%%% Sub document introduction %%%
%%%-------------------------------------------------%%%

\section{Introduction}

% Ich würde hier ohne Unterkapitel arbeiten und die Einführung einfach durch Abschnitte gliedern.

% 1.Abschnitt: Bedeutung von Ungleichheit von Soziologischer Sicht

\emph{Bedeutung von ökonomischer Ungleicheit} \\

% 2.Abschnitt Trends in inequality research
\emph{Studien zur allgemeinen Entwicklung der Ungleichheit nach unterschiedliche Paradigmen}(Armut, Reiche, Mittelschicht)  [Concerns about the hollowing of the middle class is a argument why the tax studies should be exploited to gain knowledge about the whole distribution of economic resources. Until these days tax data is only used to assess the change in top income shares.] \\
\begin{itemize}
  \item Development of inequality with LIS Data (Gornick and Jäntti 2013)
  \item ranking of inequality across high-income countries (Atkinson, Rainwater and Smeeding (1995)
   \item inequality and low income/relative poverty (Ferreira and Ravallion, 2009; Nolan and Marx, 2009; Smeeding, O’Higgins and Rainwatter, 1990; Rainwater and Smeeding 2003; Förster and Vlemickx, 2004)
  \item inequality and top income (Atkinson and Piketty, 2007; Leigh, 2009)
  \item inequality and the middle class (Estache and Leipziger, 2009; Littrell et al., 2001; Birdsall, 2010; Ravallion, 2010; Frank,2007)
\end{itemize}

% 3.Abschnitt: Studien mit Steuerdaten

\emph{Studien mit Steuerdaten}
\begin{itemize}
\item Old studies with tax data (Vilfredo Pareto, 1895, 1896; Simon Kuznets, 1953)
\item More recently revival of tax-studys with focus on top income (Piketty and Saez, 2003; Atkinson and Piketty, 2007,2010).
\end{itemize}

% 4.Abschnitt: What do we know about Switzerland?

\emph{Official Data Collection} \\

\begin{itemize}
\item  LIS (1982-2004, Survey Data from Swiss Federal Statistical Office three different surveys (recent years from “income and Consumption Survey, main result; decreasing inequality)
\item  OECD (2000 und 2004, Income and consumption survey, 2008, EU-SILC. The change of survey in 2008 is consdered as a strict break. Comparison over time not possible (OECD 2012:315)
\item  Worldbank: Nothing
\item  Federal Office of statistics (1998-2011, HABE, survey data, very detailed measurement, main result; inequality in post transfer income is constant)
\item  World income inequality Database (WIID) is more or less the same data as it is published in the LIS data
\end{itemize}
\emph{Fazit}:LIS longest time series, comparability is questionable, more recent years time series from Federal Office of statistics. Both studies use survey data, which, as we will discuss later, have certain drawbacks.\\

\emph{Studies on Switzerland} \\
Vgl. Forschungsantrag (kürzen) plus neuere Studien, Verteilungsbericht 2011 und 2012 (SGB), Avenir Suisse (2013) und Gorgas and Schaltenegger (2011) und Foellmi and Martinez (2013). \\

% Should we discus the non-scientic literature?

% What is the purpose of our paper?
The purpose of the following paper is threefold. First of all, information about economic inequality in Switzerland is sparse. We  fill this gap with official data from the Swiss Federal Tax Administration (FTA). Second, we assess the suitability of the FTA data to report inequality and its changes over time. Third, we integrate the development of swiss inequality into the international picture. We can show similarities and differences to the worldwide trends and therefore contribute to the classical theories (\cite{kuznets1955economic}, \cite{nielsen_kuznets_1997}).

